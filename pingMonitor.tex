\documentclass[a4paper,10pt,twocolumn]{article}
\title{pingMonitor.py}
\author{{\small Greg Baker -- greg.baker@ifost.org.au}}
\setlength{\parskip}{2mm}
\parindent=0mm
\begin{document}
\maketitle
\tableofcontents
\newcommand{\pingmon}[0]{\texttt{pingMonitor.py}}

\section{License}

This software is copyright \copyright Greg Baker (greg.baker@ifost.org.au),
but is licensed to anyone who wants to use it for any reason
in any way.  If you want to let me know what you do with it,  even better.
Needless to say,  it doesn't come with any warranty or support.  If it
breaks,  you get to keep all the pieces.

\section{Purpose}

This program is designed to be run from cron,  and pings a list of
computers.  If any of them are down,  it sends an email to the system
admin,  unless it has sent one in the recent past.  It also alerts
when a machine that was down comes back up again.

\section{Requirements}

\pingmon uses the Python interpreter,  it doesn't greatly matter which
version.  Anything in the last few years should work, I think.  It is
slightly Unix dependant (e.g.  ping has stdout redirected,  mail is sent
by piping through a popen to mail),  but shouldn't be too hard to fix.

\section{Configuration}

The list of machines to ping is typically stored in 
\texttt{/usr/local/etc/importantDevices}.
They can be listed as IP addresses or hostnames.  
Blank lines are ignored,  as are lines beginning with \# or ; (semicolon).


At the top of the script are the configuration options -- e.g. how
regularly to remind you about machines that are still down (at the
moment, this is every hour); how long to wait for the ping to complete
before deciding that the remote host is temporarily
unreachable (currently five seconds);
how long to wait between spawning off ping processes (it runs the
pings in parallel); who to send email to in the event of some change
in status;  how many times a machine has to be unreachable before it
is declared ``down'' (currently 5).
One day I will put this into a config file or something.

It writes into files in the directory 
\texttt{/var/imon/pingMonitor},  which is also configurable.
It stores the last recorded status (and when that was) for each device listed. 


\end{document}